\documentclass[11pt,a4,twosided,singlespacing,titlepagenumber=on]{scrreprt}

\usepackage[T1]{fontenc} % Handles accents etc better in the invisible details of the pdf output.
\usepackage[latin1]{inputenc} % May or may not be needed. Says that your *.tex file is a text file with ASCII latin1 encoding. You could use e.g. utf8 instead for easier accents etc.
\usepackage[UKenglish]{babel} % Let LaTeX know what language the text is in so it can select the correct hyphenation pattern etc

%%% American Mathematical Society packages
\usepackage{amsfonts,amssymb,amsmath,amsthm}
\usepackage{amsbsy}


\usepackage{algorithm}
\usepackage[noend]{algpseudocode}

%\usepackage{bm} % Possibly a better alternative to amsbsy for making bold typeface math.

%%% Graphics packages
\usepackage{graphicx}
%\graphicspath{{figures/}} % Useful if you have lots of images and want to keep thinks tidy by having a subfolder for images
%\usepackage{epstopdf} % If you produce your graphs as .eps files but then want to compile straight to PDF (e.g. because you are using TeXworks.) you may want to use this option. A better alternative of course would be to save all your graphs as *.pdf files from the start. Note that if you are compiling to pdf through PS/DVI then all your figures should be *.eps files and the epstopdf package should not be used.
%\usepackage{tikz} %For creating vector-graphics diagrams, flowcharts etc directly in LaTeX (takes some time to learn)
\usepackage[absolute]{textpos} % Used to position the Imperial College logo. You can comment this line and the next line out if you don't use the logo.
%\setlength{\TPHorizModule}{\paperwidth}\setlength{\TPVertModule}{\paperheight}
%\setlength{\TPHorizModule}{1cm}\setlength{\TPVertModule}{1cm}


%%% Referencing and cross-referencing
%\usepackage{color}
%\usepackage[colorlinks=false,linkcolor=red,urlcolor=cyan,citecolor=blue,breaklinks,plainpages=false,pdfpagelabels]{hyperref} % To make the hyperlinked cross-referencing visible.
\usepackage[colorlinks=false,pdfborder={0 0 0},plainpages=false,pdfpagelabels]{hyperref} % If you click on an item in the table of contents or a referenced equation/figure number, the PDF will go to the desired page. Neat isn't it?
\usepackage[round,authoryear,sort]{natbib} % Enable bibtex-based bibliography generation
%\usepackage[square,numbers,sort&compress]{natbib} % If you want numbered referencing instead of author-year style.



%%%%%%%%%%%%%%%%%%%%%%%%%%%%%%%%%%%%%
%%%%% Create or control Macros   %%%%%%%%%%%%%%%%%%%%
%%%%%%%%%%%%%%%%%%%%%%%%%%%%%%%%%%%%%

%\setcounter{secnumdepth}{3} %If you want subsubsections to be numbered
\numberwithin{equation}{chapter} % Reset equation numbers after each chapter.

%%% Theorem environments
\newtheorem{theorem}{Theorem}%[chapter]
\newtheorem{proposition}[theorem]{Proposition}%[chapter]
\newtheorem{definition}[theorem]{Definition}%[chapter]
\newtheorem{lemma}[theorem]{Lemma}%[chapter]
\newtheorem{corollary}[theorem]{Corollary}%[chapter]
%
\theoremstyle{remark}
\newtheorem{remark}[theorem]{Remark}%[chapter]
\newtheorem{example}[theorem]{Example}%[chapter]

%%% Potentially useful style changes:
%\renewcommand{\titlefont}{\normalcolor \normalfont \bfseries} %Change the title font from sans-serif to serif (the same font used for the rest of the document).
%\renewcommand*{\labelitemi}{$\bullet$} %Bullet points in the itemize environment.
%\renewcommand*{\tilde}{\widetilde} % Wider tildes
%\renewcommand*{\bar}{\overline} % Wider conjugate bars

%%% Examples of commands/macros that could be useful:
%\newcommand{\expectation}[1]{\mathbb{E}\left[ #1 \right]}
%\newcommand*{\setR}{{\mathbb R}}
%\newcommand{\commentify}[1]{} %Gives you an alternative way (other than %) to comment things out.
%%% These commands make it faster to get the correct roman font in equations: (similar to \exp, \cos, \sin etc). Alternatively yuo can always use e.g. $\mathrm{Var}$, but this is better.
%\DeclareMathOperator{\bigo}{O}
%\DeclareMathOperator{\littleo}{o}
%\DeclareMathOperator{\var}{Var}
%\DeclareMathOperator{\cov}{Cov}
%\DeclareMathOperator{\trace}{trace}
%\DeclareMathOperator{\sign}{sgn}
%\DeclareMathOperator{\rank}{rank}
%\DeclareMathOperator{\vecrm}{vec} % The \vec command already exists so you can't name this \vec.




%%%%%%%%%%%%%%%%%%%%%%%%%%%%%%%%%%%%%
%%%%% Define how to create the title page  %%%%%%%%%%%%%%%%
%%%%%%%%%%%%%%%%%%%%%%%%%%%%%%%%%%%%%
\makeatletter
\newcommand*{\supervisor}[1]{\gdef\@supervisor{#1}}
\newcommand*{\CID}[1]{\gdef\@CID{#1}}
\newcommand*{\logoimg}[1]{\gdef\@logoimg{#1}}
\renewcommand{\maketitle}{
\begin{titlepage}
\ifdefined\@logoimg
\begin{textblock*}{8cm}(1.75cm,1.75cm)
\includegraphics[width=70mm]{\@logoimg}
\end{textblock*}
\vspace*{1cm}
\else
%\vspace*{0cm}
\fi
\begin{center}
\vspace*{\stretch{0.1}}
Imperial College London\\
Derpartment of Mathematics\par
\vspace*{\stretch{1}} % This inserts vertical space and allows you to specify a relative size for the vertical spaces.
{\titlefont\Huge \@title\par} % If your title is long, you may wish to use \huge instead of \Huge.
\vspace*{\stretch{2}}
{\Large \@author \par}
\vspace*{1em}
{\large CID: \@CID \par}
\vspace*{\stretch{0.5}}
{\large Supervised by \@supervisor \par}
\vspace*{\stretch{3}}
{\Large \@date \par}
\vspace*{\stretch{1}}

\textit{This report is submitted as part requirement for the MSc Degree in Statistics at Imperial College London. It is substantially the result of my own work except where explicitly indicated in the text.
The report will be distributed to the internal and external examiners, but thereafter may not be
copied or distributed except with permission from the author.}
\vspace*{\stretch{0.1}}
\end{center}%
\end{titlepage}%
}
\makeatother


%%% And the abstract page
\renewenvironment{abstract}%
{\chapter*{Abstract}\thispagestyle{plain}}%
{\clearpage}
%%% And why not change the quote environment
\newenvironment{myquote}%
{\begin{quote}{\Large{}``}}%
{\ifhmode\unskip\fi{\Large{}''}\end{quote}}


\title{State Space Modelling for Statistical Arbitrage}
\author{Philippe~Remy}
\CID{00993306}
\supervisor{Nikolas Kantas and Yanis Kiskiras}
\date{\today}
\logoimg{Imperial__4_colour_process.jpg}

\begin{document}


\maketitle

\declaration

\begin{abstract}
This project is aimed to investigate the practical benefit of using more complex modelling than what is currently standard practice in applications related to statistical arbitrage. The underlying assets will be modelled using appropriate mean-reverting time series or state space models. In order to fit these models to real data the project will involve using advanced particle methods such as Particle Markov Chain Monte Carlo. The primary aim of the project is to assess whether using more advanced modelling and model calibration will result to better performance than simple models used often in practise. This will be illustrated in numerical examples, where the computed portfolio is used for a realistic scenario obtained by popular trading platforms. Simulations will be mainly run in Matlab, but embedding C/C++ routines may be required to speed up computations. The project is a challenging computational Statistics application to finance and is this suitable for a student with an interest in finance, very good aptitute to computing and understanding of the material in the course related to Monte Carlo methods and Time Series.
\end{abstract}
\newpage
%\chapter*{Acknowledgements}
%Thank you supervisor/friends/family/pet.
%\begin{myquote}
%Include an acknowledgement.
%\end{myquote}
%\newpage

% Automatically create a table of contents
%\renewcommand{\contentsname}{Table of Contents}
%\tableofcontents
%\newpage

% Figure and table lists if you want them.
%\cleardoublepage
%\phantomsection
%\listoffigures 
%\addcontentsline{toc}{chapter}{\listfigurename}
%\newpage
%\phantomsection
%\listoftables  
%\addcontentsline{toc}{chapter}{\listtablename}
%\newpage


%\pagestyle{headings} % uncomment this if you want headers on your pages. Google fancyhdr or look into the options of scrreprt if you want different headers.
\chapter{State Space Modelling}
\section{Bootstrap Particle Filter}

The bootstrap particle filter is an iterative method for carrying out Bayesian inference for dynamic state space (partially observed Markov process) models, sometimes also known as hidden Markov models (HMMs). Here, an unobserved Markov process, $x_0,x_1,\ldots,x_T$ governed by a transition kernel $p(x_{t+1}|x_t)$ is partially observed via some measurement model $p(y_t|x_t)$ leading to data $y_1,\ldots,y_T$. The idea is to make inference for the hidden states $x_{0:T}$ given the data $y_{1:T}$. The method is a very simple application of the importance resampling technique. At each time, t, we assume that we have a (approximate) sample from $p(x_t|y_{1:t}$) and use importance resampling to generate an approximate sample from $p(x_{t+1}|y_{1:t+1}$).

More precisely, the procedure is initialised with a sample from $x_0^k \sim p(x_0),\ k=1,\ldots,M$ with uniform normalised weights ${w'}_0^k=1/M$. Then suppose that we have a weighted sample $\{x_t^k,{w'}_t^k|k=1,\ldots,M\}$ from $p(x_t|y_{1:t}$). First generate an equally weighted sample by resampling with replacement M times to obtain $\{\tilde{x}_t^k|k=1,\ldots,M\}$ (giving an approximate random sample from $p(x_t|y_{1:t}$)). Note that each sample is independently drawn from $\sum_{i=1}^M {w'}_t^i\delta(x-x_t^i)$. Next propagate each particle forward according to the Markov process model by sampling $x_{t+1}^k\sim p(x_{t+1}|\tilde{x}_t^k),\ k=1,\ldots,M$ (giving an approximate random sample from $p(x_{t+1}|y_{1:t}$)). Then for each of the new particles, compute a weight $w_{t+1}^k=p(y_{t+1}|x_{t+1}^k$), and then a normalised weight ${w'}_{t+1}^k=w_{t+1}^k/\sum_i w_{t+1}^i$.

It is clear from our understanding of importance resampling that these weights are appropriate for representing a sample from $p(x_{t+1}|y_{1:t+1})$, and so the particles and weights can be propagated forward to the next time point. It is also clear that the average weight at each time gives an estimate of the marginal likelihood of the current data point given the data so far. So we define

$$\displaystyle \hat{p}(y_t|y_{1:t-1})=\frac{1}{M}\sum_{k=1}^M w_t^k$$

and

$$\displaystyle \hat{p}(y_{1:T}) = \hat{p}(y_1)\prod_{t=2}^T \hat{p}(y_t|y_{1:t-1}).$$

Again, from our understanding of importance resampling, it should be reasonably clear that $\hat{p}(y_{1:T})$ is a consistent estimator of ${p}(y_{1:T}$). It is much less clear, but nevertheless true that this estimator is also unbiased. The standard reference for this fact is Del Moral (2004), but this is a rather technical monograph. A much more accessible proof (for a very general particle filter) is given in Pitt et al (2011).

It should therefore be clear that if one is interested in developing MCMC algorithms for state space models, one can use a pseudo-marginal MCMC scheme, substituting in \hat{p}_\theta(y_{1:T}) from a bootstrap particle filter in place of $p(y_{1:T}|\theta)$. This turns out to be a simple special case of the particle marginal Metropolis-Hastings (PMMH) algorithm described in Andreiu et al (2010). 

In bootstrap particle filter, $\pi(x_k^{(i)} | x_{0:k-1}^{(i)}, y_{0:k}) = p(x_k^{(i)}|x_{k-1}^{(i)})$. When the transition prior probability is used as the importance function, the weights update formula is simplified:

$$
w_k^{(i)} = w_{k-1}^{(i)} \frac{p(y_k|x_k^{(i)})p(x_k^{(i)}|x^{(i)}_{k-1})}{\pi(x_k^{(i)}|x^{(i)}_{0:k-1},y_{0:k})}= w_{k-1}^{(i)} p(y_k|x_k^{(i)})
$$ 

In the bootstrap particle filter, it is clear that the average weight at each time gives an estimate of the marginal likelihood of the current data point given the data so far:

$$ p^N_{\theta}(y_t | y_{0:t-1}) = \frac{1}{N} \sum_{k=1}^N w_t^k$$
The marginal likelihood at time $T$ is:
$$ p_{\theta}(y_{0:T}) = p(y_0)\prod_{t=1}^T p(y_t | y_{1:t-1})$$

Again, from our understanding of importance resampling, it should be reasonably clear that $\hat{p}^N_{\theta}(y_{0:T})$ is a consistent estimator of $p_{\theta}(y_{0:T})$. It is much less clear, but nevertheless true that this estimator is also unbiased according to Del Moral (2004).

The marginal log likelihood and its estimator are:
\begin{align*}
\log(p_{\theta}(y_{0:t})) &= \log(p(y_0)) + \sum_{t=1}^t \log \left(p(y_t | y_{1:t-1}) \right) \\
\log(p^N_{\theta}(y_{0:t})) &= - \log(N) + \sum_{t=1}^t \log \left(\frac{1}{N} \sum_{k=1}^N w_t^k \right)
\end{align*}

\begin{algorithm}
\caption{Bootstrap Particle Filtering Algorithm (SIR)}\label{euclid}
\begin{algorithmic}[1]
\Procedure{}{}
\\
\For{i from 1 to N} 
	\State $x_k^{(i)} \sim \pi(x_k | x_{0:k-1}^{(i)}, y_{0:k})$
	\State $w_k^{(i)} = \hat{w}_k^{(i-1)} p(y_k | x_k^{(i)})$
\EndFor{end}
\\
\For{i from 1 to N} 
	\State $w_k^{(i)} = \hat{w}_k^{(i)} / \sum_{j=1}^N \hat{w}_k^{(j)} $
\EndFor{end}
\\
\State $x_k$ = resampling($x_k$, $w_k)$
\For{i from 1 to N} 
	\State $w_k^{(i)} = 1 / N$
\EndFor{end}
\\
\EndProcedure
\Return $(x_k, w_k)$

\end{algorithmic}
\end{algorithm}

\newpage
\section{Stochastic Volatility}
In this section, we introduce the standard stochastic volatility with Gaussian errors. Next, we consider different well-known extensions of the SV model. The first extension is a SV model with Student-t errors. In the second extension, we incorporate a leverage effect by modeling a correlation parameter between measurement and state errors. In the third extension, we implement a model that has both stochastic volatility and moving average errors.

\subsection{Simple SV Model}
The standard discrete-time stochastic volatility model for the returns $Y_n$ is defined as:
\begin{align*}
  X_{n+1} &=  \rho X_{n} + \sigma V_n \\
  Y_n &=  \beta \exp \left( \frac{X_n}{2} \right) W_n
\end{align*}
where  $\{V_n\}$,$\{W_n\}$ are two independent sequences of independent standard normal random variables. Let $\theta = (\rho, \sigma^2, \beta)$. Notice that the non-linearity of the models relies in the non-additive noise of the transition Kernel. $X_n$ is the unobserved log-volatility associated to the observed log-returns $Y_n$, $\sigma$ is the volatility of the log-volatility and $\rho$ is the persistence parameter. The condition $|\rho| < 1$  is imposed to have a stationary process with the initial condition $X_0 \sim \mathcal{N} \left(0, \frac{\sigma^2}{1-\rho^2} \right)$.

\subsection{SVt - Student-t innovations}
The first extension is a stochastic volatility model with ${W_n} \sim St(\nu)$ where $St$ stands for the Student-t distribution with $\nu > 2$. The conditional density becomes (pdf variable substitution):

$$p(y_n | x_n, Y_{n-1},  \theta) = \frac{\Gamma(\frac{\nu+1}{2})}{\Gamma(\frac{\nu}{2}) \sqrt{(\nu-2)\pi}} \frac{1}{\sigma_n}\left( 1 + \frac{y_n^2}{\sigma_n^2 \nu}\right)^{-\frac{v+1}{2}}$$
where $\sigma_n = \beta \exp \left(\frac{x_n}{2} \right)$. We then follow the sampling steps as before.

\subsection{SVL - Stochastic Volatility Leverage}
In the second extension, we incorporate a leverage effect by letting $c$ denote the correlation between $V_n$ and $W_n$. Here, we use the fact that $V_n = c W_n + \sqrt{1-c^2} \Psi_n$ where $\Psi_n \sim N(0,1) :$

\begin{align*}
X_{n+1} &=  \rho X_{n} + \sigma \left(c W_n + \sqrt{1-c^2} B_n \right) \\
X_{n+1} &=  \rho X_{n} + \sigma \left(c Y_n \exp \left( - \frac{X_n}{2} - \log(\beta) \right)  + \sqrt{1-c^2} B_n \right)
\end{align*}

Notice that we need to sample the additional parameter $c$.

\subsection{SV-MA(1) - Moving Average}
We can also expand the plain stochastic volatility model by allowing the errors in the measurement equation to follow a moving average (MA) process of order $m$. This means that the errors in the measurement equation are no longer serially independent as for the plain SV model. Here, we choose a more simple specification and set $m = 1$. Hence, our model becomes:

\begin{align*}
Y_n &=  \beta \exp \left( \frac{X_n}{2} \right) W_n + \Psi \beta \exp \left( \frac{X_{n-1}}{2} \right) W_{n-1} \\
X_{n+1} &=  \rho X_{n} + \sigma V_n
\end{align*}
We ensure that the root of the characteristic polynomial assocaited with the MA coefficient $\Psi$ is outside the unit circle, $|\Psi$| \\

In the following, we will assume that a process $(X_t)_{t \in \mathbb{N}}$ is adapted to a filtration $(\mathcal{F}_t)_{t \in \mathbb{N}}$ which presents the accrual of information over time. We denote by $\mathcal{F}_t = \sigma \{X_s : s \leq t \}$ the $\sigma$-algebra generated by the history of $X$ up to time $t$. The corresponding filtration is then called the natural filtration.

$$Var(y_t | \mathcal{F}_{t-1}) = \exp(X_t) + \Psi^2 \exp(X_{t-1})$$
because $X_t$ is measurable with regard to $\mathcal{F}_{t-1}$. It turns out that the conditional variance of $y_t$ is varying through two channels. Estimating this model is straightfoward as again we only need to make small adjustments in the codes.

\subsection{SV-M}
Let's consider the population stochastic volatility in mean (SVM) model where $\exp(X_t /2)$ appears in both the conditional mean and the conditional volatility. We follow the same notation as before and define the SVM model as: 

$$y_t = \beta \exp\left(\frac{X_t}{2}\right) + \exp\left(\frac{X_t}{2}\right) + W_t, \text{ }, W_t \sim N(0,1)$$
where $X_t$ is ruled by the dynamics of a simple SV model. The conditional probability density of $y_t$ is $p(y_t | x_t, Y_{t-1}, \theta) \sim N(\beta \exp(x_t/2), exp(x_t))$.

\subsection{TFSV - Two Factors}
Finally, we estimate a two factor SV model. It is defined as:
\begin{align*}
  X_{n+1} &=  \rho_1 X_{n} + \sigma_2 V_n, \text{ } |\rho_1| < 1, V_n \sim N(0,1)\\
  Z_{n+1} &=  \rho_2 Z_{n} + \sigma_2 P_n, \text{ } |\rho_2| < 1, P_n \sim N(0,1) \\
  Y_n &=  \exp \left(\frac{\mu}{2} + \frac{X_n+Z_n}{2} \right) W_n, \text{ } |\rho_2| < 1, W_n \sim N(0,1)
\end{align*}
$\theta$ is enriched with the new parameters. Thus, we only need to modify the particle filter such that we drawa two sets of particles (one for $X_t$ and one for $Z_t$) instead of one.

\section{Model Comparison}
The output of the particle filter is an estimate of $p(y|\theta)$, with the unobserved states integrated out. The marginal likelihood for a model $\mathcal{M}$ is defined as:

$$p(Y_T | \mathcal{M}) = \int_{\theta} p(Y_T | \theta, \mathcal{M}) p(\theta | \mathcal{M}) d\theta$$

Gelfand and Dey (1994) proposed a very general estimate for this marginal likelihood:

$$\frac{1}{N} \sum_{i=1}^N \frac{g(\theta_i)}{p(Y_T | \theta_i) p(\theta_i)} \rightarrow \frac{1}{p(Y_T)}  \text{ as It}_{mcmc} \rightarrow +\infty$$
For this estimator to be consistent, $g(\theta_i)}$ must be thin-tailed relative to the denominator. For most cases, a multivariate Normal distribution $N(\theta^*, \Sigma^*)$ can be used, where $\theta^* = \frac{1}{N} \sum_{i=1}^N \theta^i$ and $\Sigma^* = \frac{1}{N-1} \sum_{i=1}^N \left(\theta^i - \theta^*\right)\left(\theta^i - \theta^*\right)^T$. The difficulty of this approach resides in the implementation. As a matter of fact, $p(Y_T | \theta)$ is usually very small as $T$ grows. The trick here is to consider the sum of the exponential of the logarithms and factorize by the maximum logarithm to avoid rounding errors. For example, when N = 3 and let assume that the log-terms on the LHS are equal to $-120$, $-121$ and $-122$ :

\begin{align*}
p(Y_T)^{-1} &= e^{-120} + e^{-121} + e^{-122} \\
- \log p(Y_T) &= \log (e^{-120} ( 1 + e^{-1} + e^{-2})) \\
 \log p(Y_T) &= 120 - \log ( 1 + e^{-1} + e^{-2})) \simeq 119.6
\end{align*}


When $p(Y_T | \mathcal{M_A})$ and $p(Y_T | \mathcal{M_B})$ have been estimated, Kass and Raftery (1995) suggest to use twice the logarithm of the Bayes factor for model comparison, $2 \log BF_{\mathcal{M_{AB}}}$. The evidence of $\mathcal{M_A}$ over $\mathcal{M_B}$ is based on a rule-of-thumb: 0 to 2 not worth more than a bare mention, 2 to 6 positive, 6 to 10 strong, and greater than 10 as very strong.

\section{Resampling}
Resampling is a key component of the Particle Filter. Different methods exist: stratified, systematic and residuals resampling. In practical applications, they are generally found to provide comparable results. Despite the lack of complete theoretical analysis of its behavior, systematic resampling is often preferred because it is the simplest method to implement. Randal Douc proved that residual and stratified resampling methods dominate the basic multinomial approach, in the sense of having lower conditional variance for all configurations of the weights.

\begin{table}[h]
\centering
\label{table1}
\begin{tabular}{|l|l|l|l|l|}
\hline
Resampling method     & Residual & Stratified & Systematic & Multinomial \\ \hline
Time (in seconds) & 18.90    & 0.62       & 0.63       & 1.87        \\ \hline
\end{tabular}
\caption{Time spent to resample 100K times 1000 weights}
\end{table}
The multinomial implementation is the MATLAB default version. According to Randal Douc and the performance results, the stratified resampling seems the most compelling method to use inside the particle filters. This part is critical because it can represent up to 50$\%$ of the total time spent in the filter.

\section{PMMH}

In a more general context, a Metropolis Hastings scheme can be used to target $p(\theta| y)$ with the ratio:

\begin{align*}
\min \left( 1, \frac{p(\theta^\star)}{p(\theta)} \times  \frac{q(\theta|\theta^\star)}{q(\theta^\star|\theta)} \times \frac{p({y}|\theta^\star)}{p({y}|\theta)} \right)
\end{align*}

where $q(\theta^\star|\theta)$ is the proposal density. In Hidden Markov Models, the marginal likelihood $p(y|\theta)$ is often intractable and the ratio is hard to compute. The simple likelihood-free scheme targets the full joint posterior $p(\theta,x|y)$. Usually $p(x|\theta)$ is tractable. For instance, in the linear Gaussian case, $x_{0:T}$ can be simulated when $\rho$ and $\tau$ are known. The MH is built in two stages. First, a new $\theta^*$ is proposed from $q(\theta^\star|\theta)$ and then, $x^*$ is sampled from $p(x^\star|\theta^\star)$. The pair $(\theta^\star,x^\star)$ is accepted with the ratio:

\begin{align*}
\min \left( 1, \frac{p(\theta^\star)}{p(\theta)} \times  \frac{q(\theta|\theta^\star)}{q(\theta^\star|\theta)} \times \frac{p(y|{x}^\star,\theta^\star)}{p(y|{x},\theta)} \right)
\end{align*}
At each step, the $x^*$ is consistent of $\theta^*$ thanks to this proposed mecanism. The main drawback is that $T$ must be really small to have a good acceptance rate. As a matter of fact, since the conditional likelihood $p(y|{x}^\star,\theta^\star)$ is a product of $T$ terms over the path $x_{0:T}$, it becomes intractable very quickly as $T$ increases. The sampled $x^*$ must also be consistent with $y$. This is the reason why $x^*$ should be sampled from $p(x^\star|\theta^\star,y)$. The ratio becomes:

\begin{align*}
 \min \left(1, \frac{p(\theta^\star)}{p(\theta)}   \frac{p({x}^\star|\theta^\star)}{p({x}|\theta)}   \frac{f(\theta|\theta^\star)}{f(\theta^\star|\theta)}   \frac{p(y|{x}^\star,\theta^\star)}{p(y|{x},\theta)}  \frac{p({x}|y,\theta)}{p({x}^\star|y,\theta^\star)} \right)
\end{align*}

Using the basic marginal likelihood identity of Chib (1995), the ratio is simplified to:

\begin{align*}
 \min \left(1, \frac{p(\theta^\star)}{p(\theta)}  \frac{p(y|\theta^\star)}{p(y|\theta)} \frac{f(\theta|\theta^\star)}{f(\theta^\star|\theta)} \right)
\end{align*}
Remarkably $x$ is no more present and the ratio is exactly the same as the marginal scheme shown before. Indeed the ideal marginal scheme corresponds to PMMH when $N \rightarrow +\infty$. The likelihood-free scheme is obtained with just one particle in the filter. When $N$ is intermediate, the PMMH algorithm is a trade-off between the ideal and the likelihood-free schemes, but is always likelihood-free when one bootstrap particle filter is used.

\cleardoublepage
\phantomsection
\addcontentsline{toc}{chapter}{\bibname} % Add an entry for the Bibliography in the Table of Contents
%\pagestyle{headings}
\bibliographystyle{abbrvnat} % set the bibliography style
\bibliography{bibtexfile} % generate the bibliography
%\bibliographystyle{abbrvnat} % <- Mistake in earlier version. After the bibliography is created it's too late to change the style.
% Pick a sensible bibliography style. 
% If like above you choose to use author-year style citations, you must choose a natbib-compatible bibliographystyle such as 'plainnat'. Abbrvnat is one of the few built-in styles of this type. You may find many more bibliography styles (*.bst files) online. You can also choose to write your bibliography manually instead of using BibTeX (This will take you longer, unless you plan to use the content of a BibTeX-generated *.bbl file as your starting point).
% When creating a BibTeX file using e.g. reference manager software, note that the entries may contain additional unwanted fields which you would then need to erase. For example, you probably don't want to include the URL of a journal paper in your bibliography.


%\cleardoublepage \fancyhead[L]{APPENDIX}
\appendix % This line declares that you are starting the appendix.
% If you want a single Appendix and want it to be called Appendix instead of Appendix A, the following should work:
%\setcounter{secnumdepth}{-1} %This turns off automatic chapter numbering
%\chapter{Appendix}


\end{document}